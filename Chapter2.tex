\chapter{Tools and Technologies Used}
\section{HTML}
Hypertext Markup Language (HTML) is the standard markup language for creating web pages and web applications. With Cascading Style Sheets (CSS) and JavaScript it forms a triad of cornerstone technologies for the World Wide Web. Web browsers receive HTML documents from a web server or from local storage and render them into multimedia web pages. HTML describes the structure of a web page semantically and originally included cues for the appearance of the document.
HTML elements are the building blocks of HTML pages. With HTML constructs, images and other objects, such as interactive forms, may be embedded into the rendered page. It provides a means to create structured documents by denoting structural semantics for text such as headings, paragraphs, lists, links, quotes and other items. HTML elements are delineated by tags, written using angle brackets. Tags such as <img /> and <input /> introduce content into the page directly. Others such as <p>...</p> surround and provide information about document text and may include other tags as sub-elements. Browsers do not display the HTML tags, but use them to interpret the content of the page.
HTML can embed programs written in a scripting language such as JavaScript which affect the behavior and content of web pages. Inclusion of CSS defines the look and layout of content. 
\section{Bootstrap-A CSS Framework}
Cascading Style Sheets (CSS) is a style sheet language used for describing the presentation of a document written in a markup language. Although most often used to set the visual style of web pages and user interfaces written in HTML and XHTML, the language can be applied to any XML document, including plain XML, SVG and XUL, and is applicable to rendering in speech, or on other media. Along with HTML and JavaScript, CSS is a cornerstone technology used by most websites to create visually engaging webpages, user interfaces for web applications, and user interfaces for many mobile applications.

CSS is designed primarily to enable the separation of presentation and content, including aspects such as the layout, colors, and fonts. This separation can improve content accessibility, provide more flexibility and control in the specification of presentation characteristics, enable multiple HTML pages to share formatting by specifying the relevant CSS in a separate .css file, and reduce complexity and repetition in the structural content.

Bootstrap is a free and open-source front-end web framework used for  designing websites and web applications. It contains HTML- and CSS-based design templates for typography, forms, buttons, navigation and other interface components, as well as optional JavaScript extensions. Unlike many web frameworks, it concerns itself with front-end development only.

Bootstrap is the second most-starred project on GitHub, with more than 111,600 stars and 51,500 forks.
\section{Javascript}
JavaScript, often abbreviated as JS, is a high-level, interpreted programming language. It is a language which is also characterized as dynamic, weakly typed, prototype-based and multi-paradigm.

Alongside HTML and CSS, JavaScript is one of the three core technologies of the World Wide Web. JavaScript enables interactive web pages and this is an essential part of web applications. The vast majority of websites use it, and all major web browsers have a dedicated JavaScript engine to execute it.

As a multi-paradigm language, JavaScript supports event-driven, functional, and imperative (including object-oriented and prototype-based) programming styles. It has an API for working with text, arrays, dates, regular expressions, and basic manipulation of the DOM, but the language itself does not include any I/O, such as networking, storage, or graphics facilities, relying for these upon the host environment in which it is embedded.
\section{PHP}
PHP (Hypertext Preprocessor) is a server-side scripting language that is commonly used to create dynamic web pages. It is open-source software, which means that it can be freely modified and distributed. PHP is often used in conjunction with a web server like Apache and a database management system like MySQL, to create dynamic web applications.

PHP code can be embedded within HTML code, making it easy to create interactive and dynamic web pages. Some of the features of PHP include:
Variables: PHP supports a wide range of data types, including integers, strings, and arrays.
Functions: PHP has a large number of built-in functions that can be used to perform a wide range of tasks, such as working with strings, arrays, and dates.
Control Structures: PHP supports all of the standard control structures, such as if/else statements and loops.
Error handling: PHP has built-in error handling capabilities, which can be used to handle errors and exceptions in a consistent and controlled manner.
Database connectivity: PHP has built-in support for connecting to a wide range of databases, such as MySQL, Oracle, and Microsoft SQL Server.
Security: PHP provides several built-in functions and extensions to help secure your code from common security threats such as SQL injection, cross-site scripting (XSS), and cross-site request forgery (CSRF).

\section{MySQL}
MySQL is a widely used, open-source relational database management system (RDBMS). It is designed to store and manage large amounts of data in an efficient and organized manner. MySQL is particularly popular for web-based applications, due to its speed, reliability, and ease of use.
Some of the features of MySQL include:
SQL support: MySQL uses the SQL (Structured Query Language) to interact with the database, allowing developers to easily create, read, update, and delete data.
Tables and relationships: MySQL allows you to create multiple tables, each with its own structure and data, and to define relationships between them.
Indexes: MySQL allows you to create indexes on specific columns in a table to improve query performance.
Stored procedures and triggers: MySQL allows you to create stored procedures and triggers to automate repetitive tasks and enforce data integrity.
Security: MySQL supports a variety of security features, such as user management, encryption, and access control, to help protect data from unauthorized access.

%%********************Chapter 3**********
