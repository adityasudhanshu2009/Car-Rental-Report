\chapter{Introduction to Database Management System}
\section{Introduction}
Databases and database technology have a major impact on the growing use of computers. It is fair to say that databases play a critical role in almost all areas where computers are used, including business, electronic commerce, engineering, medicine, genetics, law, education, and library science. The word database is so commonly used that we must begin by defining what a database is. Our initial definition is quite general. A database is a collection of related data. By data, we mean known facts that can be recorded and that have implicit meaning. For example, consider the names, telephone numbers, and addresses of the people you know. You may have recorded this data in an indexed address book or you may have stored it on a hard drive, using a personal computer and software such as Microsoft Access or Excel. This collection of related data with an implicit meaning is a database. The preceding definition of database is quite general; for example, we may consider the collection of words that make up this page of text to be related data and hence to constitute a database. However, the common use of the term database is usually more restricted. 
A database has the following implicit properties:

\begin{itemize}
\item{A database represents some aspect of the real world, sometimes called the miniworld or the universe of discourse (UoD). Changes to the miniworld are reflected in the database.}
\item{A database is a logically coherent collection of data with some inherent meaning. A random assortment of data cannot correctly be referred to as a database}
\item{A database is designed, built, and populated with data for a specific purpose. It has an intended group of users and some preconceived applications in which these users are interested}
\end{itemize}
A database management system (DBMS) is a collection of programs that enables users to create and maintain a database. The DBMS is a general-purpose software system that facilitates the processes of defining, constructing, manipulating, and sharing databases among various users and applications. Defining a database involves specifying the data types, structures, and constraints of the data to be stored in the database. The database definition or descriptive information is also stored by the DBMS in the form of a database catalog or dictionary; it is called meta-data. Constructing the database is the process of storing the data on some storage medium that is controlled by the DBMS. Manipulating a database includes functions such as querying the database to retrieve specific data, updating the database to reflect changes in the miniworld, and generating reports from the data. Sharing a database allows multiple users and programs to access the database simultaneously.

\thispagestyle{fancy}


\section{ History of DBMS}
The history of database management systems (DBMS) can be traced back to the 1960s, when IBM developed the Integrated Data Store (IDS) for the U.S. Air Force. This was one of the first database management systems to be developed and used in a production environment.

In the 1970s, the relational model for databases was introduced by Dr. E.F. Codd at IBM. He proposed a new way of organizing data using tables with rows and columns, and a structured query language (SQL) for data manipulation. This model became the basis for the development of relational database management systems (RDBMS), such as IBM's System R and later, Oracle.

In the 1980s, the popularity of RDBMSs grew as personal computers became more powerful and affordable. This led to the development of several new RDBMSs, such as SQL Server, MySQL, and PostgreSQL.

In the 1990s, the internet and the World Wide Web emerged as a major new platform for information management and communication. This led to the development of new types of DBMSs, such as document databases and key-value stores, which are better suited for storing and retrieving unstructured data.

In recent years, NoSQL databases have gained popularity for their ability to handle large amount of unstructured data and also for their scalability, performance and ease of handling distributed data.

Overall, the history of DBMSs has been one of constant evolution, driven by advances in technology, changes in the way data is used and stored, and the emergence of new platforms and use cases.

\section{Applications of DBMS}
Applications where we use Database Management Systems are: \\
\begin{itemize}
\item \textbf{Telecom:} There is a database to keeps track of the information regarding calls made, network usage, customer details etc. Without the database systems it is hard to
maintain that huge amount of data that keeps updating every millisecond.
\item \textbf{Industry:} Where it is a manufacturing unit, warehouse or distribution centre, each one needs a database to keep the records of ins and outs. For example distribution
centre should keep a track of the product units that supplied into the centre as well as
the products that got delivered out from the distribution centre on each day; this is
where DBMS comes into picture.
\item \textbf{Banking System: } For storing customer info, tracking day to day credit and debit
transactions, generating bank statements etc. All this work has been done with the help
of Database management systems.
\item \textbf{Education Sector: } Database systems are frequently used in schools and colleges to store and retrieve the data regarding student details, staff details, course details, exam
details, payroll data, attendance details, fees details etc. There is a hell lot amount of
inter-related data that needs to be stored and retrieved in an efficient manner.
\item \textbf{Online Shopping: }You must be aware of the online shopping websites such as
Amazon, Flip kart etc. These sites store the product information, your addresses and
preferences, credit details and provide you the relevant list of products based on your
query. All this involves a Database management system.
\end{itemize}

